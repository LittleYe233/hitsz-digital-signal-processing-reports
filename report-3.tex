\documentclass{dspreport}

%%
%% User settings
%%

\classno{\blank{  }}
\stuno{\blank{  }}
\groupno{\blank{  }}
\stuname{\blank{  }}
\expdate{\lxgw\expdatefmt\today}
\expidx{三}
\expname{有限脉冲响应滤波器}

%%
%% Document body
%%

\begin{document}
% First page
% Some titles and personal information are defined in ``\maketitle''.
\maketitle

\section{实验记录与问题思考}
3.5.2 使用 2 单位延迟 FIR 的陷波滤波器

(3) 幅频响应曲线测试.

利用 Lab 14 选项卡中的示波器上观测 300Hz-3000Hz 频率范围内的信号幅度, 适当地选择 10 个频率点, 第 6 个点位于陷波滤波器的幅值最低点, 其余的各点位于陷波器最低点的两侧, 并用 MATLAB 作图.
\begin{block}

\end{block}

\begin{figure}[H]
    \centering
\end{figure}

\begin{block}

\end{block}

\textbf{问题 1} 在幅值响应图最低点处, 测量陷波的频率为\underline{\blank{  }}, 以及相对深度为\underline{\blank{  }}.

\textbf{问题 2} 采样频率为 10kHz, 单位圆上 180 度点处的对应的频率是多少?
\begin{block}

\end{block}

\textbf{问题 3} 分别降低 $b_1$ 和 $b_2$ 值, 陷波频率和陷波深度应该是减小还是增加.
\begin{block}

\end{block}

\textbf{(选做) 问题 4} 利用一元二次方程求根公式, 解释步骤 (5) 中改变系数引起零点变化的程度, 可通过求导获得信号的变化率 (斜率), 对衰减速率进行估计.
\begin{block}

\end{block}

3.5.3 利用陷波滤波器消除单频干扰并恢复有用信号

(1) 使用台式信号发生器作为信号 $f_1$ 和 $f_2$ 的源, $f_1$ 和 $f_2$ 分别为 500Hz 和 1300Hz 的正弦波. 正弦波的幅值分别为\underline{\blank{  }}和\underline{\blank{  }}.

(2) 分别在时域和频域中观察采样保持器的输出信号以及陷波滤波器的输出信号, 并\textbf{截图保存}, 说明是否可以用 3.5.2 中的陷波滤波器抑制信号 $f_2$.
\begin{block}

\end{block}

\textbf{问题 5} 在频域上观测, 信号 $f_2$ (1300Hz) 在陷波频率处衰减了多少 dB?
\begin{block}

\end{block}

\textbf{问题 6} 根据本实验已经得到的结果, 需要如何更改增益 $b_1$ 来降低陷波频率?
\begin{block}

\end{block}

\textbf{问题 7} z 平面上, 零点-原点连线后与坐标横轴所形成的夹角 $\theta$, 零点所对应的陷波频率 $f_0$, 采样时钟频率 $f_s$, 三者之间的关系方程式是什么?
\begin{block}

\end{block}

(5) 将陷波滤波器的输出接入到可调谐低通滤波器中, 以恢复有效信号; 观察所恢复正弦波的 ``干净'' 程度, 并\textbf{截图}保存; \textbf{测量此时低通滤波器的通带, 截止频率}, 分别为\underline{\blank{  }}和\underline{\blank{  }}.
\begin{block}

\end{block}

\begin{figure}[H]
    \centering
\end{figure}

\begin{block}

\end{block}

(6) 将 $f_1$ (500Hz) 信号看作干扰信号, $f_2$ (1300Hz) 信号看作有效信号. 更改 $b_1$ 的值, 尽可能地消除双音信号的干扰频率分量, \textbf{截图}保存陷波滤波器滤波后的图像.
\begin{block}

\end{block}

\begin{figure}[H]
    \centering
\end{figure}

\begin{block}

\end{block}

\textbf{问题 8} 尽可能保持有效信号频率分量 $f_2$ (1300Hz) 不衰减或有较少衰减, 当 $b_1$ 的值是多少时, 实验能获得最大程度对干扰分量的衰减? 得的衰减是多少 dB?
\begin{block}

\end{block}

\textbf{问题 9} $f_1$ (500Hz) 信号为干扰信号, $f_2$ (1300Hz) 信号为有效信号情况下, 用现有的陷波滤波器和低通滤波器能否恢复有效信号? 若不能请解释原因.
\begin{block}

\end{block}

\section{MATLAB 仿真}
\textbf{(注意需提交 Matlab 源代码及运行结果)}

(1) 利用最初设置的加法器增益作为系统差分方程的系数 (即 $b_0=1.0$; $b_1=-1.3$; $b_2=0.902$), 利用 MATLAB 仿真在 300-3000Hz 频率范围内画出该滤波器的幅频响应特性曲线, 在同一张图中对比仿真结果和实际测试结果.
\begin{block}

\end{block}

\begin{figure}[H]
    \centering
\end{figure}

\begin{block}

\end{block}

(2) 设计一个低通滤波器 (可利用 MATLAB 中的 Filter Designer), 要求通带截止频率和阻带截止频率与 3.5.3 (5) 所用到的低通滤波器的幅频特性一致, 并用 MATLAB 画出其频谱图.
\begin{block}

\end{block}

\begin{figure}[H]
    \centering
\end{figure}

\begin{block}

\end{block}

(选做) (3) 利用上两步中设计的陷波滤波器和低通滤波器, 在 MATLAB 中复现 3.5.3 中 (1), (2), (5) 实验过程.

\textbf{要求:} $f_1$ (500Hz) 信号看作有用信号, $f_2$ (1300Hz) 信号看作干扰信号, 分别画出采样保持器输出端, 陷波滤波器输出端及可调谐低通滤波器输出端三个端口信号的时域波形图和对应频谱图, 并在实验报告中与实验测试结果对比.
\begin{block}

\end{block}

\begin{figure}[H]
    \centering
\end{figure}

\begin{block}

\end{block}

\section{实验体会与建议}
\begin{block}

\end{block}

\end{document}
